%! Author = hmaier
%! Date = 09.09.21

% Preamble
\documentclass[12pt,german,ngerman]{report}
\title{Künstliche Intelligenz - Ein Blick in die Zukunft?}
\author{Hendrik Maier}
\date{}


% Packages
\usepackage[T1]{fontenc} % für die spezielle Quotierung, die mit
\usepackage{german}
\usepackage{titling}
\usepackage{titlesec}
% zeilenabstand
\usepackage[onehalfspacing]{setspace}


% quotation-marks
\usepackage[
    left = \flqq{},% the special quote on the left (opening)
    right = \frqq{},% the special quote on the right (closing)
            ]{dirtytalk} % quoting
\usepackage{csquotes}

% bibliography
%\usepackage[backend=biber,autocite=footnote,style=authortitle-ibid,babel=other]{biblatex}
\usepackage[
    backend=biber,
   % cite=footnote,
    style=numeric,
]{biblatex}
\addbibresource{ai-references.bib}

% less blank space infornt of chapters
\titleformat{\chapter}[display]
{\normalfont\huge\bfseries}{\chaptertitlename\ \thechapter}{20pt}{\Huge}

% this alters "before" spacing (the second length argument) to 0
\titlespacing*{\chapter}{0pt}{0pt}{30pt}


% Document
\begin{document}

    \begin{titlepage}
        \centering
        {\scshape\LARGE Leibniz Institut für Pflanzenbiochemie\par}
        \vspace{1cm}
        {\scshape\Large Fachbericht September 2021\par}
        \vspace{1.5cm}
        {\huge\bfseries Künstliche Intelligenz - Ein Blick in die Zukunft?\par}
        \vspace{2cm}
        {\Large Abteilung: Geräte \& IT-Service\par}
        \vspace{1.5cm}
        {\Large\itshape Hendrik Maier\par}
        \vfill

% Bottom of the page
        % {\large \today\par}
    \end{titlepage}

    \tableofcontents
    \newpage

% Begin of the Text
\chapter{Was ist eigentlich künstliche Intelligenz - Definition und Einleitung}
    

\chapter{Der Traum vom mechanischen Helferlein - Geschichte}

Wie auch bei vielen anderen neuzeitlichen Erfindungen, spielt Kultur in Form von Literatur eine entscheidende Rollen
in der KI-Forschung. Erst durch die Vorstellungskraft von Autoren wie Isaac Asimov\footnote{bekannt durch: Die Foundation-Trilogie, derZweihunderjährige Mann}
oder Jule Verne\footnote{bekannt durch: Zwanzig Tausend Meilen unter dem Meer, Die Reise zum Mittelpunkt der Erde}, wurde Generationen
von Forschenden inspiriert die Grenzen des Möglichen auszutesten. Die Idee der KI ist dabei schon so alt wie unsere
Zivilisation. Schon der griechische Dichter Homer schrieb von mechanischen Dienern die den Göttern bei ihrem Mahl
Wein nachschenkten.\cite[53]{buchanan2005very} Auch wenn diese Verwendung eines Robotors aus unserer heutigen
industriellen Zeit eher banal erscheint, war ein solcher Apparat zu damaligen Zeiten visionär.
Nicht weniger visionär, war die Vorstellung des Philosophen und Naturforschers Gottfried Wilhelm Leibniz der mechanische
Richter imaginierte, die aufgrund von logischen Regeln Rechtsfälle zwischen Parteien aushandeln.\cite[53]{buchanan2005very}
Vergleicht man diese Vorstellungen mit der heutigen Zeit, sieht man das sich die Wünsche der Menschen durch die Zeit hindurch
nicht groß verändert haben. Leibniz und Homer beschreiben beide dienende und logisch operiende Maschinen.
Doch gestehen beide ihren erdachten Apparaten eine wesentliche Fähigkeit nicht ein: die Denkfähigkeit. \\

Um dies zu wagen braucht es Mitte des 20. Jahrhunderts erst den britischen Mathematiker Alan Turing, 
der mit seinem Gedankenexperiement, dem \say{Nachnahmungs-Spiel}, die Frage nach intelligenten und denkenden Maschinen stellte.
Dieses auch als \say{Turing-Test} bekannte, Gedankenexperiement stellt seit dem einen Richtwert für den Forschritt der KI-Forschung.\footnote{Mehr dazu im Kapitel 5, Gedankenexperimente}
Diese entstand zur gleichen Zeit als Ergebnis der Verbesserung der Halbleiter-Technologie, die als Basis für 
moderne binäre Computersysteme gilt. Die dadurch vergrößerte Rechenkapazität und -leistung eröffnete bisher nicht realisierbare Forschungsmöglichkeiten. Seit dem finden KI-Forschende immer mehr Anwendungsmöglichkeiten um die Grenzen der KI immer weiter auszuweiten.\\

1997 erzeugt das Schreiben und Test von KI-Programmen, erstmals einen großes Medienecho, als der von IBM programmierte Schachcomputer
\say{Deep Blue} den damals amtierenden Schach-Weltmeister Gary Kasparov, schlug.\cite{chessbase2017kasparovdeepblue} Die Komplexität von Schach war mit purer brutaler
Rechnenleistung bisher noch nicht geknackt wurden. Dazu bedarf es erst einem \say{Verständnis} von Schach dass bis zu diesem
Zeitpunkt nur Menschen zugänglich war. Das Programm probierte also nicht mehr nur alle Züge aus, sondern prognostizierte die 
Sinnvollsten. Damit wurde eine Zeitenwende eingeleitet und der breiten Öffentlichkeit die Macht von KI demonstriert.
Das dadurch erschaffene Interesse beflügelte die KI-Forschung und weitete diese in mehr Anwendungsbereiche aus.
In 2021 hatten ein Großteil der Menschen in ihrem Alltag schonmal Kontakt zu KI-Technologie. 
Ob bei der Suche im Internet oder bei Nutzung einer Sprachsteuerung, steht im Hintergrund immer diese Technologie.
Die Vorstellung von einem mechanischen Helferlein weicht allmälich dem Bild der allwissenden künstlichen Intelligenz.
Prognosen über den weiteren Forschritt der KI-Forschung werden meistens eher optimistisch bewertet.\footnote{Im fünften Kapitel wird auch eine pessimistische Sichtweise behandelt.} 
Eine der bekanntesten Hypothesen wurde 1993 vom US-Amerikanischen Mathematiker und Informatiker Vernor Vinge aufgestellt und 2003
erneut bestätigt. Vinge bespreibt ein spätestens 2030 auftretendes Ereignis, welches er die \say{technologische Singularität}\cite[1]{vinge1993technological} nennt. Er nennt mehrere Szenarien welche direkt oder indirekt, ein Entstehen einer denkfähigen 
superintelligente Entität beschreiben, die auf Computertechnologie basiert.\\

Betrachtet man Vinges Prognose für die nahe Zukunft\footnote{oder allein schon den heutigen Stand der Technologie mit Siri, Google und Co.} und Homers Beschreibung von Wein ausschenkenden Dienern,
wird ein Entwicklungsprozess sichtbar, dem jeglicher Vergleich fehlt. 



\chapter{Schwache versus Starke - Arten von KI}
    \section{Schwache KI}
        \subsection{}
    \section{Starke KI}
        \subsection{Technolgische Singularität}
        \subsection{künstliche Ethik}

\chapter{Technische Grundlagen}
    \section{Wie erschafft man eine künstliche Intelligenz?}
    
        \subsection{Wie arbeitet eine künstliche Intelligenz bzw. wie kann man Erkenntnis gewinnen?}

    \section{Top-Down Methode (Deduktiv)}
        \subsection{Supervised Learning}
        \subsection{Pre-Labeled (Test-und Training-) Data}
        \subsection{Specific Application Area}
        \subsection{Pro \& Contra}
        
    \section{Bottom-Up (Induktiv)}
        \subsection{Unsupervised Learning}
        \subsection{Unlabeled Training Data (no Test, no Accuary)}
        \subsection{Wide Range of Output}
        \subsection{Pro \& Contra}

    \section{Deep Learning - Neurale Netzwerke}
        \section{Implementierung eines menschlichen Gehirn mithilfe von unsupervised learning}

\chapter{Gedankenexperimente}
    \section{The Imitation Game - Das Nachahmungsspiel}
    \section{Das Chinesische Zimmer}

\chapter{Schlussfolgerung}
    \section{Unterschied Mensch und Maschine}
        \subsection{Menschen lernen sich selbst besser kennen wenn sie KI erfoschen}
        \subsection{Was Maschinen vom Menschen unterscheidet}

    \printbibliography[title={Quellenverzeichnis}]
\end{document}
