%! Author = hmaier
%! Date = 09.09.21

% Preamble
\documentclass[12pt,german,ngerman]{report}
\title{Künstliche Intelligenz - Ein Blick in die Zukunft?}
\author{Hendrik Maier}
\date{}


% Packages
\usepackage[T1]{fontenc} % für die spezielle Quotierung, die mit
\usepackage{german}
\usepackage{titling}
\usepackage{titlesec}


% quotation-marks
\usepackage[
    left = \flqq{},% the special quote on the left (opening)
    right = \frqq{},% the special quote on the right (opening)
            ]{dirtytalk} % quoting
\usepackage{csquotes}

% bibliography
%\usepackage[backend=biber,autocite=footnote,style=authortitle-ibid,babel=other]{biblatex}
\usepackage[
    backend=biber,
   % cite=footnote,
    style=numeric,
]{biblatex}
\addbibresource{ai-references.bib}

% zeilenabstand
\usepackage[onehalfspacing]{setspace}

% Document
\begin{document}

    \begin{titlepage}
        \centering
        {\scshape\LARGE Leibniz Institut für Pflanzenbiochemie\par}
        \vspace{1cm}
        {\scshape\Large Fachbericht September 2021\par}
        \vspace{1.5cm}
        {\huge\bfseries Künstliche Intelligenz - Ein Blick in die Zukunft?\par}
        \vspace{2cm}
        {\Large Abteilung: Geräte \& IT-Service\par}
        \vspace{1.5cm}
        {\Large\itshape Hendrik Maier\par}
        \vfill

% Bottom of the page
        % {\large \today\par}
    \end{titlepage}

    \tableofcontents
    \newpage

% Begin of the Text
\chapter{Was ist eigentlich \say{künstliche Intelligenz} - Definition und Einleitung}

\chapter{Der Traum vom mechanischen Helferlein - Geschichte}

\chapter{Schwache versus Starke - Arten von KI}


\chapter{Technische Grundlagen}
    \section{Wie erschafft man eine künstliche Intelligenz?}
        \subsection{Wie arbeitet eine künstliche Intelligenz bzw. wie kann man Erkenntnis gewinnen?}

    \section{Top-Down Methode (Deduktiv)}
        \subsection{Supervised Learning}
        \subsection{Pre-Labeled (Test-und Training-) Data}
        \subsection{Specific Application Area}
        \subsection{Pro \& Contra}
        
    \section{Bottom-Up (Induktiv)}
        \subsection{Unsupervised Learning}
        \subsection{Unlabeled Training Data (no Test, no Accuary)}
        \subsection{Wide Range of Output}
        \subsection{Pro \& Contra}

    \section{Deep Learning - Neurale Netzwerke}
        \section{Implementierung eines menschlichen Gehirn mithilfe von unsupervised learning}

\chapter{Gedankenexperimente}
    \section{The Imitation Game - Das Nachahmungsspiel}
    \section{Das Chinesische Zimmer}

\chapter{Schlussfolgerung}
    \section{Unterschied Mensch und Maschine}
        \subsection{Menschen lernen sich selbst besser kennen}
        \subsection{Was Maschinen vom Menschen unterscheidet}
    \section{Schwierigkeiten bei der Entwicklung}
        \subsection{Ethik schwer zu implementieren}

    \printbibliography[title={Quellenverzeichnis}]
\end{document}
