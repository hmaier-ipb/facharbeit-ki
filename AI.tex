%! Author = hmaier
%! Date = 09.09.21

% Preamble
\documentclass[12pt]{report}
\title{Künstliche Intelligenz - Ein Blick in die Zukunft}
\author{Hendrik Maier}
\date{}


% Packages
\usepackage[T1]{fontenc} % für die spezielle Quotierung, die mit
\usepackage{german}
\usepackage{titling}

% quotation-marks
\usepackage[
    left = \flqq{},% the special quote on the left (opening)
    right = \frqq{},% the special quote on the right (opening)
            ]{dirtytalk} % quoting
\usepackage{csquotes}

% bibliography
\usepackage{biblatex}
\addbibresource{ai_references.bib}



% Document
\begin{document}
    \maketitle

    \tableofcontents
    \newpage

    \section{Einführung: Was ist eigentlich Künstliche Intelligenz?}
    Im allgemeinen Sprachgebrauch beschreibt der Begriff \say{Künstliche Intelligenz} die Fähigkeit von Computer-Programmen,
    individulle Problemstellungen zu lösen. Die geschieht ohne die direkte Hilfe eines Technikers oder Programmieres.
    Mit \say{Intelligenz} ist dabei speziell gemeint, das vorher nie bearbeitete Probleme mithilfe ihrer Daten
    trotz ihrer Fremdheit zum Lösen eines Problems verwendet werden können. Die Fähigkeit der \say{Künstlichen Intelligenz}
    wird mithilfe des maschinellen Lernens erreicht. Grob gesagt kriegt ein Algorithmus (welcher als Computer-Programm geschrieben ist),
    eine Vielzahl von Daten eingespielt. Durch die Menge der Daten gelingt es dem Algorithmus, ein spezifisches
    Modell zu erstellen welches auf ähnliche aber jedoch unbekannte Daten angewendet werden kann.
    
    \section{Geschichte}

    \section{Starke versus Schwache KI}
    Die Idee eines mechanischen Helfer, der logische zu bearbeitende Aufgaben übernimmt, ist gar nicht so neu wie man
    zuerst vermuten würde. Wie auch andere bahnbrechende Erfindungen, werden die ersten Schritte auch bei dieser Idee
    mit einem Blatt Papier und etwas Tinte gegangen. Unter anderem Isaac Asimov hatte die Idee
    eines Roboters der sowohl als mechanischer Diener als auch als selbstdenkender Künstler agieren kann.\cite{asimov2000der}
    Mit dieser Idee, die nicht nur eine logisch agierende Maschine vorsieht, sondern auch ein denkendes Individuum, macht Asimov
    eine Teilung in zwei Kategorien die bis heute gilt. Die Rede ist von schwacher und Künstlicher Intelligenz.\\

    Als \emph{schwache Künstliche Intelligenz} bezeichnet man ein Großteil der heute eingesetzten Programme, die mit
    maschinellen Lernenn trainiert worden sind. Diese Art der KI erfüllt einen ganz bestimmten Zweck:
    beispielweise Spracherkennung, bei Amazons Alexa und Apples Siri. Beide Programme sind mit einer Vielzahl von
    Datensätzen trainiert worden die Sprache repräsentieren. Dabei muss vom Menschen unter anderem festgelegt werden
    was genau verwendetbare Daten sind (und nicht etwa externe Störgeräusche) und welche Sprache
    (beispielweise Englisch oder Deutsch) erkannt werden soll. Natürlich sind dies nur zwei Faktoren die der Computer zum
    Lernen benötigt, reale Computerprogramme benötigen tausende Parameter um sich an verschiedene Gegebenheiten anzupassen.
    Essenzielle Vorraussetzung das die Faktoren genutzt werden können ist dass der Mensch der Maschine
    das Ziel gibt, sich mit den vorgegebenen Daten zu beschäftigen. Weshalb und wofür die Daten benutzt werden, spielt
    dabei für die Maschine keine Rolle. Endprodukt (tech. \say{Modell}) der Beschäftigung sind Regeln und Zusammenhänge
    mit denen logische Probleme bearbeitet werden können. Ohne die Zuarbeit des Menschens, ist dieses Endprodukt nicht
    möglich, was bedeutet dass andere Probleme auf Grundlage der bisher eingepflegen Daten nicht zu lösen sind.
    Eine schwache KI kann also bestimmte Problemstellungen lösen, und dies sogar mit hoher Effizienz, doch bei
    unbekannten Parametern, versagen gelernte Regeln und Zusammenhänge.\\

    Um mit






    \section{Einsatzgebiete und Anwendungsfälle}

    \subsection{Anwendungsbeispiele IPB}
    \section{Technische Grundlagen}
    \subsection{Supervised Learing}
    \subsection{Unsupervised Learning}
    \subsection{Linear Regression}
    \subsection{Logistic Regression}
    \subsection{Decision Tree}
    \subsection{Random Rorest Model}
    \subsection{Neural Networks (Deep Learning)}
    \section{Philosophische Betrachtung}
    In diesem abschließenden Abschnitt werde ich das Thema zusammenfassen sowie den Versuch unternehmen,
    Denkanstöße zu geben.
    \subsection{Intelligenzbegriff}
    In der Menschheitsgeschichte hatten Betrachtende bisher nur Tiere, Pflanzen oder andere Menschen zum Vergleich
    um den Begriff der \say{Intelligenz} zu definieren. Mit der vorranschreitenden Integration von Systemen die maschinelles
    Lernen nutzen um menschen-gemachte Aufgaben zu bewältigen, muss neu geklärt werden:
        \begin{displayquote}
            Was genau ist \say{Intelligenz}?
            Ab wann ist ein Wesen \say{intelligent}?
            Wie geht man mit weiter
        \end{displayquote}

    \printbibliography
\end{document}