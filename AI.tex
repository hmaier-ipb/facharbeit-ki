%! Author = hmaier
%! Date = 09.09.21

% Preamble
\documentclass[12pt]{report}
\title{Künstliche Intelligenz - Ein Blick in die Zukunft}
\author{Hendrik Maier}
\date{}


% Packages
\usepackage[T1]{fontenc} % für die spezielle Quotierung, die mit
\usepackage{german}
\usepackage{titling}

% quotation-marks
\usepackage[
    left = \flqq{},% the special quote on the left (opening)
    right = \frqq{},% the special quote on the right (opening)
            ]{dirtytalk} % quoting
\usepackage{csquotes}

% bibliography
\usepackage{biblatex}
\addbibresource{ai_references.bib}



% Document
\begin{document}
    \maketitle

    \tableofcontents
    \newpage

    \section{Einführung: Was ist eigentlich Künstliche Intelligenz}
    Im allgemeinen Sprachgebrauch beschreibt der Begriff \say{Künstliche Intelligenz} die Fähigkeit von Computer-Programmen,
    individulle Problemstellungen zu lösen. Die geschieht ohne die direkte Hilfe eines Technikers oder Programmieres.
    Mit \say{Intelligenz} ist dabei speziell gemeint, das vorher nie bearbeitete Probleme mithilfe ihrer Daten
    trotz ihrer Fremdheit zum Lösen eines Problems verwendet werden können. Die Fähigkeit der \say{Künstlichen Intelligenz}
    wird mithilfe des maschinellen Lernens erreicht. Grob gesagt kriegt ein Algorithmus (welcher als Computer-Programm geschrieben ist),
    eine Vielzahl von Daten eingespielt. Durch die Menge der Daten gelingt es dem Algorithmus, ein spezifisches
    Modell zu erstellen welches auf ähnliche aber jedoch unbekannte Daten angewendet werden kann.
    Man könnte eine Maschine


    \section{Einsatzgebiete und Anwendungsfälle}

    \subsection{Anwendungsbeispiele IPB}
    \section{Technische Grundlagen}
    \subsection{Supervised Learing}
    \subsection{Unsupervised Learning}
    \subsection{Linear Regression}
    \subsection{Logistic Regression}
    \subsection{Decision Tree}
    \subsection{Random Rorest Model}
    \subsection{Neural Networks (Deep Learning)}
    \section{Philosophische Betrachtung}
    In diesem abschließenden Abschnitt werde ich das Thema zusammenfassen sowie den Versuch unternehmen,
    Denkanstöße zu geben.
    \subsection{Intelligenzbegriff}
    In der Menschheitsgeschichte hatten Betrachtende bisher nur Tiere, Pflanzen oder andere Menschen zum Vergleich
    um den Begriff der \say{Intelligenz} zu definieren. Mit der vorranschreitenden Integration von Systemen die maschinelles
    Lernen nutzen um menschen-gemachte Aufgaben zu bewältigen, muss neu geklärt werden:
        \begin{displayquote}
            Was genau ist \say{Intelligenz}?
            Ab wann ist ein Wesen \say{intelligent}?
            Wie geht man mit weiter
        \end{displayquote}

    \printbibliography
\end{document}