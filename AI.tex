%! Author = hmaier
%! Date = 09.09.21

% Preamble
\documentclass[a4paper,12pt,german,ngerman]{report}
\title{Künstliche Intelligenz - Ein Blick in die Zukunft?}
\author{Hendrik Maier}
\date{}


% Packages
\usepackage[T1]{fontenc} % für die spezielle Quotierung, die mit
\usepackage{german}
\usepackage{titling}
\usepackage{titlesec}
\usepackage{hyperref} % for cross-referencing

% quotation-marks
\usepackage[
    left = \flqq{},% the special quote on the left (opening)
    right = \frqq{},% the special quote on the right (opening)
            ]{dirtytalk} % quoting
\usepackage{csquotes}

% bibliography
\usepackage[
    backend=biber,
% cite=footnote,
    style=numeric,
]{biblatex}
\addbibresource{ai-references.bib}

% zeilenabstand
\usepackage[onehalfspacing]{setspace} %set spacing between the lines

\titleformat{\chapter}[display]
{\normalfont\huge\bfseries}{\chaptertitlename\ \thechapter}{20pt}{\Huge}

% this alters "before" spacing (the second length argument) to 0
\titlespacing*{\chapter}{0pt}{0pt}{30pt}

% Document
\begin{document}

    \begin{titlepage}
        \centering
        {\scshape\LARGE Leibniz Institut für Pflanzenbiochemie\par}
        \vspace{1cm}
        {\scshape\Large Fachbericht September 2021\par}
        \vspace{1.5cm}
        {\huge\bfseries Künstliche Intelligenz - Ein Blick in die Zukunft?\par}
        \vspace{2cm}
        {\Large Abteilung: Geräte \& IT-Service\par}
        \vspace{1.5cm}
        {\Large\itshape Hendrik Maier\par}
        \vfill

% Bottom of the page
        % {\large \today\par}
    \end{titlepage}

    \tableofcontents
    \newpage

    \chapter{Was ist Künstliche Intelligenz?}
    In den letzten Jahren hat der Begriff der \say{Künstlichen Intelligenz} (KI) als Schlagwort ein erhebliches Gewicht
    erlangt. Verschiedene Medien berichten immer wieder von der KI als einer Art magischen Technologie, die unsere
    Art wie wir Leben komplett revolutionieren wird. KI soll schon in vielen Technologien intergriert sein, die wir heute
    benutzen und soll es in wenigen Jahren noch mehr sein. Da stellen sich dem alltäglichen Menschen schon
    verschiedene Fragen wie: was ist KI, wie funktioniert sie und warum ist sie so faszinierend?\\

    Bevor ich auf diese wichtigen Fragen eingehe, möchte ich vorerst den Begriff der \say{Künstlichen Intelligenz}
    erläutern, um Klarheit um diesen oft genutzten und doch unklaren Begriff zu schaffen. \say{Künstliche Intelligenz} ist ein Kunstbegriff der erstmals
    1956 auf der \emph{Darthmouth Conference}\cite[57]{buchanan2005very} benutzt wurde.
    Der Begriff \emph{Intelligenz} kommt vom lateinischen \say{intelligere} was soviel wie
    Einsehen, Begreifen und Erkennen bedeutet.\cite{piaget2000psychologie}
    \say{Künstlich} verweist in diesem Zusammenhang auf die unnatürliche Herkunft dieser Intelligenz.
    Da Erkenntnis\cite{duden2021erkenntnis} und somit Intelligenz, immer etwas Geistiges ist, könnnte
    man KI wie folgt definieren:
    \begin{displayquote}
        \say{Künstliche Intelligenz} beschreibt einen nicht-menschlichen/unnatürlichen Geist,
        der die Fähigkeit besitzt, durchs eigenständige Denken Erkenntnis zu erlangen.
    \end{displayquote}
    Dies ist, wenn man sich an der Bedeutung der Wortes orientiert, eine Idealbeschreibung künstlicher Intelligenz.
    Da dies noch lange nicht erreicht ist, könnte man eine realere Definition schaffen mit der der aktuelle
    Stand der Technologie besser abgebildet wird:
    \begin{displayquote}
        Eine \say{Künstliche Intelligenz} ist Computerprogramm das mit der Analyse von Statistiken Vorraussagen treffen kann.
    \end{displayquote}
    Obwohl bei Definitionen sich augenscheinlich starkt unterscheiden, kann man die zweite Definition als Fundament für
    die erste Definition sehen. Um einen denkfähigen Geist zu erschaffen bedarf es Erfahrungen, die das menschliche Gehirn lehren
    Situationen vorrauszusehen.\footnote{Evolutionär gesehen ist der menschliche Geist und die damit verbundenen Vorhersagen über die Realität
    das mächtigste Werkzeug im Kampf ums Überleben.}
    Computer haben keine Sinne die sie mit der äußeren Welt verbinden und müssen daher erst mit Statistiken gefüttert werden um Vorraussagen zu geben.
    Einem Computer die gleiche Fähigkeit wie man selbst zu geben, macht wahrscheinlich ein Großteil der Faszination
    beim Entwickeln einer KI aus. Um dies zu bewerkstelligen ist Forschung in den unterschiedlichsten Feldern nötig.
    Das Produkt dieser Forschung, welches in der ersten Definition beschrieben wird, wird wahrscheinlich Folgen auf
    die menschliche Zivilisation haben, wie die Erfindung des Feuers.\\

    Mit diesem Fachbericht werde ich versuchen einen kleinen Überblick zu diesem scheinbar riesigen Thema zu geben.
    Ziel ist es die Idee und die Technologie von künstlicher Intelligenz so zu vermitteln, dass es für einen Laien greifbarer wird.


    \chapter{Geschichte der KI - Der Traum vom mechanischen Helferlein}
    Wie bei vielen neuzeitlichen Erfindungen wurde auch die Forschung an \say{Künstliche Intelligenz}
    erstmalig von verschiedenen Denkern und Schriftstellern angestoßen.
    Nicht erst Science-Fiction Autoren wie Isaac Asimov oder Jule Vernes haben die Idee von intelligenten Maschinen
    entwickelt, sondern schon der Grieche Homer schrieb von mechanischen Dienern die den Göttern beim Abendsessen
    Wein nachschenkten.\cite[53]{buchanan2005very} Auch wenn diese Verwendung von KI aus unserem heutigen Standpunkt
    eher banal erscheint, ist ein solcher Apparat zu damaligen Zeiten undenkbar.
    Ein wenig weiter dachte der Philosoph Gottfried Wilhelm Leibniz, der über mechanische Richter nachdachte
    die aufgrund von logischen Regeln Rechtsfälle aushandeln.\cite[53]{buchanan2005very}
    Dieses Beispiel stößt schon ziemlich nah an die Vorstellung von künstlicher Intelligenz die wir heutzutage haben.
    Was beide Beispiele jedoch gemeinsam haben ist dass keiner von beiden ihren Apparaten eigenes Denken
    gibt. Sie werden lediglich als logisch operierende Maschinen angesehen, die ohne den Menschen nicht wissen würden
    was sie tun sollten.
    Um die Möglichkeit in Erwägung zu ziehen ob Maschinen denken könnten,
    brauchte es Mitte des 20. Jahrhunderts erst den Mathematiker Alan Turing.\cite{sesink1993menschliche}
    Turing entwickelte das \say{Nachahmungs-Spiel}, welches als der \say{Turing Test} bekannt geworden ist.
    Mithilfe dem sich vergrößernden Speicherplatz und der höheren Geschwindigkeit von Speichern und Prozessor,
    wurde es in den 1950er und 60er Jahren möglich, erste Programm zu schreiben die den \say{Turing Test} bestreiten sollten.
    Das Schreiben und Testen verschiedener Computerprogramme gipfelte erstmals 1997 in dem Schach-Spiel des Progamms \emph{Deep Blue}
    gegen den Schach-Weltmeister Gary Kasparov.\cite{hsu1999ibm}
    Wichtig zu Erkennen, wenn man die Entwicklungsgeschichte der Künstlichen Intelligenz betrachtet, ist dass es Fortschritt verschiedener
    wissenschaftlicher Perspektiven\cite{buchanan2005very} bedurfte, um zur modernen Idee der \say{Künstlichen Intelligenz} zu gelangen.
    Dazu gehören Disziplinen wie Biologie, Logik und Philosophie, Maschinenbau und Psychologie.\cite[56]{buchanan2005very}
    Alle diese Felder der Wissenschaft werden unter anderem auch wenn der Motivation angetrieben, herrauszufinden was genau
    das menschliche Bewusstsein oder auch die menschliche Intelligenz ist. Der Mensch muss also erst herrausfinden was
    sein eigener Geist bzw. sein Bewusstsein ist, bevor er diesen in eine Maschine implementieren kann die die Fähigkeit
    hat, ihn zu imitieren. Falls diese Suche einmal erfolgreich ist, wird dies nicht nur ein mechanisches Helferein
    sein sondern ein ebenbürtiger Partner, der den Menschen unterstützt, ergänzt und wahrscheinlich sogar ersetzt.


    \chapter{Arten der Künstlichen Intelligenz - Starke versus Schwache KI}
    Die Idee eines mechanischen Helfer, der logische zu bearbeitende Aufgaben übernimmt, ist gar nicht so neu wie man
    zuerst vermuten würde. Wie auch andere bahnbrechende Erfindungen, werden die ersten Schritte auch bei dieser Idee
    mit einem Blatt Papier und etwas Tinte gegangen. Isaac Asimov hatte in seinem Science-Fiction Roman \say{Der 200-Jährige Mann}
    die Idee eines Roboters der sowohl als mechanischer Diener als auch als selbstdenkender Künstler agieren kann.\cite{asimov2000der}
    Mit dieser Idee, die nicht nur eine logisch agierende Maschine vorsieht, sondern auch ein denkendes Individuum, macht Asimov
    eine Teilung in zwei Kategorien die bis heute gilt. Die Rede ist von schwacher (logisch agierender) und starker (denkender) Künstlicher Intelligenz.\\

    \section{Schwache KI}
    Als \emph{schwache Künstliche Intelligenz} bezeichnet man ein Großteil der heute eingesetzten Programme, die mit
    maschinellen Lernen trainiert worden sind.\cite{ibm2021whatisAI} Diese Art der KI erfüllt vordefinierte Aufgaben,
    wie beispielweise die Erkennung von Sprache oder Objekten. Dafür wird ein Vielzahl von vorbearbeiteten Beispielen der KI
    zum Lernen gegeben. Diese Beispiele sind vom Menschen auf eine Art und Weise bearbeitet so dass sie auf ein spezielles Ziel hindeuten.
    Der Mensch gibt der Maschine also ein Ziel so dass sie sich mit den vorgegebenen Daten beschäftigen kann.
    Ohne vorbestimmtes Ziel wäre es der Maschine nicht möglich die Daten zu deuten und zu verarbeiten.
    Endprodukt (tech. \say{Modell}) der Beschäftigung mit den Daten sind Regeln und Zusammenhänge
    mit denen die Problemstellung bearbeitet werden können. Ohne die Zuarbeit des Menschens, ist dieses Endprodukt nicht
    möglich, was bedeutet dass andere Probleme auf Grundlage der bisher eingepflegen Daten nicht zu lösen sind.
    Eine schwache KI kann also bestimmte trainierte Problemstellungen lösen, und dies sogar mit hoher Effizienz und Genauigkeit, doch bei
    unbekannten Parametern, versagen gelernte Regeln und Zusammenhänge.\\

    \section{Starke KI}
    Um fremde unspezifizierte Problemstellungen zu Lösen, benötigt es einer \emph{starken Künstlichen Intelligenz}.
    Diese erweiterte Form der KI ist zum derzeitgen Zeitpunkt \footnote{Ende 2021} noch nicht realisiert worden und lässt sich am
    einfachsten mithilfe des \say{Turing Tests} definieren.\footnote{Der Turing wird in Kapitel "Gedankenexperiemente, ausführlich beschrieben}
    Dieser Test wurde Mitte des 20. Jahrhunderts von Turing, einem britischen Mathematiker, erdacht und bespreibt folgendes Spiel:
    \begin{displayquote}
        Ein Mensch und ein Fragesteller werden in zwei seperierte
        Räume aufgeteilt. Ein Fragesteller, der keinen Sichtkontakt zu jeweils zu einem noch zum anderen der beiden Räume hat
        muss durchs Fragen herausfinden, wer von beiden der Mensch und wer der Computer ist.
        Ziel des Computers ist den Fragenden irrezuleiten, so dass er glaubt dass der Computer der Mensch ist.
        Ziel der befragten Person ist es dem Fragenden bei der Identfikaton der Maschine zu helfen.\cite{turing1950computing}
    \end{displayquote}
    Falls es dem Computer gelingt, den Fragenden irrezuleiten und ihn (den Computer) als Person zu identifieren, hat der Computer
    den \say{Turing Test} bestanden und gilt somit als denkfähigen Wesen was als \emph{starke Künstliche Intelligenz}
    bezeichnet wird.\cite{oppy&dowe2020turingtest}. Ein solches denkfähiges Wesen besitzt die Fähigkeit verschiedene Problemstellungen
    auf kreative Art und Weise zu lösen, da es nicht wie ein klassischer Computer fest auf ein Thema trainiert ist sondern
    sich flexibel selber(!) Gedanken machen kann.\\

    Hier zeigt sich nun auch der genaue Unterschied zwischen schwacher und starker Künstlicher Intelligenz: eine Machine,
    die auf Grund gelernter Regeln Probleme lösen kann ist \emph{schwach}. Eine Maschine die jedoch ebenfalls selbst denken
    kann ist \emph{stark}.

    \chapter{Technische Grundlagen}
    \section{Wie funktioniert eine künstliche Intelligenz?}
        Wie schon im ersten Kapitel grob beschrieben worden ist, ist eine künstliche Intelligenz nicht anders als ein
        Computerprogramm welches aufgrund von analysierten Statistiken Vorraussagen trifft. Dieses Vorraussagen kann
        nur am Ende eines Prozesses stattfinden, den man als \say{maschinelles Lernen} bezeichnet. Maschinelles Lernen
        oder auch ML ist die Technik die hinter künstlicher Intelligenz steht. Was bei einem Auto der Motor ist,
        ist bei künstlicher Intelligenz das maschinelle Lernen. Ziel des maschinellen Lernens ist immer das generieren
        eines Modell, auch \say{Machine-Learning-Model} genannt. Mit diesem Modell kann man dann am Ende Vorraussagen treffen.
        Um ein Modell dieser Art komplett zu sehen und zu begreifen, ist ziemlich viel Mathematik nötig. Da mir nicht
        genügend Zeit zur Verfügung steht um dies voll und ganz auszuarbeiten, werden ich ein praktischeren Pfad nehmen
        und vorprogrammierte Bibliotheken nutzen um die Vorhergehensweise deduktive \footnote{Deduktion: vom Allgemeinen zum Einzelnen \cite{deduktion2021unileipzig}}
        zu erläutern.

    \section{Datasets}
        Zum maschinellen Lernen braucht man im ersten Schritt Trainingsdaten. Diese Trainingsdaten müssen numerische
        Repräsentation von bestimmten Objekten sein. Bilder, Audiodateien oder auch Text können verwendet werden,
        solange es möglich ist, diese in Zahlen umzuwandeln. Diese Trainingsdaten oder im Fachjargon auch \say{Datasets}
        genannt, sind das Fundament jeder künstlicher Intelligenz, da diese aus ihnen Vergleichswerte ziehen kann. Die Anzahl
        der zu vergleichenden Objekte ist dabei von entscheidender Bedeutung. Als Faustregel gilt: je mehr desto besser.\footnote{Vergleich Dataset: 1000 Datenpunkte = wenig/schwach, 100.000 Datenpunkte = groß/stark}

    \section{Maschinelles Lernen}
        Um aus einem Dataset ein Modell zu gewinnen gibt es zwei Methoden. Die erste Methode ist beaufsichtigtes Lernen (supervised learning) und
        die zweite Methode ist unbeaufsichtigtes Lernen (unsupervised learning).

    \section{Beaufsichtigtes Lernen (Supervised Learning) }
        Beim beaufsichtigten Lernen müssen alle Daten gekennzeichnet sein.
        Er selber definiert die Parameter die zum maschinellen Lernen notwenig sind.
        Diese Parameter setzen sich aus einem Ziel, Trainingsdaten und Testdaten zusammen.
        Das Ziel richtet sich nach dem Problem das gelöst werden soll. Dies kann entweder ein
        Klassifikations- oder ein Regressionsproblem sein.\cite{kibuisness2021supervised}
        Falls man eine Unterscheidung sucht, zum Beispiel in zwei oder mehrere Kategorien, spricht
        man von einem Klassifikationsproblem. Sucht man allerdings Zusammenhänge zwischen verschiedenen
        augenscheinlich fremden Variablen, spricht man von einem Regressionsproblem. Je nach Problem gibt
        es verschiedene Algorithmen die auf ein Dataset angewendet werden können. Das durch die Alogrithmen
        berechnete Modell, kann als Graph visulisiert werden. In diesem werden alle Datenpunkte abgebildet und von
        einer Linie durchzogen, die entweder als unterscheidende Grenze oder als Vorhersage agiert.\footnote{Hier Grafik einfügen!}
        Um diese Datenpunkte zu unterscheiden, müssen die einzelnen Datenpunkte klar gekennzeichnet sein.
        Bei manchen Datensets dessen Datenpunkte sich zu schwach oder zu stark unterscheiden, kann es sein das man mit dem Phänomen des
        \say{Over- oder Underfitting} konfrontiert wird. Dann ist das Modell unbrauchbar, da keine nützlichen Vorhersagen getroffen werden können.
        Um dies zu prüfen und gegebenenfalls dagegenzuwirken, teilt man das Datenset in Trainingsdaten und Testdaten.
        Als Trainingsdate bezeichnet man einen Großteil der Daten die zum anlernen des Modells verwendet werden. Die Testdaten
        machen dabei einen Bruchteil aus und werden im Nachhinein auf das Modell angewandt um die Varianz zu testen.
        Das bedeutet herrauszufinden wie repräsentativ die Daten für eine anwendung auf fremde Daten sind.
        Ist die Varianz zu hoch, besteht Overfitting. Die Trainings- und Testdaten unterscheiden sich nicht genug um
        mit fremden Daten arbeiten zu können. Sind die Daten zu unterschiedliche und kann kein Muster festgestellt werden
        besteht Underfitting, die Varianz ist zu niedrige und es können ebenfalls keine fremden Daten verarbeitet werden.
        Nach all diesen Schritten wird ebenfalls klar welche Art der künstlichen Intelligenz damit erreicht werden kann:
        eine schwache. Um eine starke KI zu erreichen, bräuchte es unendlich viel Konfiguration des in Menschens in verschiedenen
        Bereichen des Lebens um ein allheitliches Bild zu gewinnen. Schon allein bei der Erkennung eines
        Apfels müssten beispielweise, tausende Bilder von Äpfeln mit einem Quadrat um den Apfel gekennzeichnet sein. Nur so
        kann der Algorithmus die zum Apfel dazugehörigen Pixel einordnen.

    \section{Unbeaufsichtigtes Lernen (Unsupervised Learning) }
        Ein Gegenentwurf, bzw. eine Weiterentwicklung des maschinellen Lernens ist das \say{Unbeausichtigte Lernen}.
        Hauptunterschied zum beaufsichtigten Lernen ist das beim unbeaufsichtigten Lernen jegliche Kennzeichnung der Datenpunkte fehlt.
        Der Mensch spielt also höchste eine Rolle bei der Programmierung des künstlichen Intelligenz, jedoch nicht beim
        im Lernprozess, da dieser komplett autark ab abläuft. Wie auch beim supervised Learning gibt es hier auch verschiedene
        Algorithmen um Problemstellungen zu bearbeiten. Einer der fortschrittlichsten Algorithmen ist das \say{Deep Learning}.
        Diese Art des maschinellen Lernens basiert auf dem Modell von neuronalen Netzwerken, die das menschliche
        Gehirn nachempfinden.\cite[]{ibm2021deeplearning}
        Wie schon beschrieben, ist es nicht nötig die eingegebenen Daten zu kennzeichnen. Dem Algorithmus werden also
        meist Rohe Daten geliefert aus dem dann ein Modell entstehen soll. Beim Deep Learning repliziert der Algorithmus
        die eingegebenen Daten und vergleicht dann Eingabe und Ausgabe. Durchs vergleichen der unterschiedlichen Daten
        entsteht ein Modell, das von alleine neue unbekannte Daten klassifizieren kann.



   % \chapter{einsatzgebiete und Anwendungsfälle}
   % aufgrund des großen Potenzial welches in vorherigen Kapitel beschrieben worden ist, scheinen
   % die grenzen künstlicher Intelligenz schier unendlich. Um dieses riesige Feld einzugrenzen,
   % welches sich durch die gesamte menschliche Kultur zieht, werde ich im folgenden mehrere
   % situationen, von schwacher sowie starker KI, beschreiben die sich im IPB anwenden lassen könnten.

   % \section{effiziente Budget-Ausschöpfung im Einkauf}
   % \section{}
   % \section{}
   % \section{}

    \chapter{Gedankenexperimente}

    \section{Turing Test - Das Nachahmungs-Spiel (The Imitatione Game) }
    Mitte des 20. Jahrhunderts stellte der britische Mathematiker Alan Turing als erster die Frage ob Maschinen denken könnten.
    Diese Frage gilt seit dem nun als Kernfrage in der KI-Forschung. Um diese Frage zu beantworten schlägt Turing das
    \say{Imitation Game} oder auch \say{Nachahmungs-Spiel}, als Test für die Denkfähigkeit von Maschinen vor.
    Als Maschine schlägt er explizit einen digitalen oder elektronischen Computer vor und schließt
    biologische Möglichkeiten, wie einen aus einer Zelle gezüchteten Menschen komplett aus.\cite[435]{turing1950computing}
    Eine vollständige Abkapselung einer digitalen Maschine, als eigenständiges Gerät ist jedoch nicht möglich, das sie auf dem
    Fundament von menschlichen Prinzipien konstruiert worden ist. Maschinen müssen immer als menschgemacht gedacht werden.
    Ob sie auf diesem Hintergrund denkfähig sein können, wird durch das folgende Spiel herrausgefordert:
    \begin{displayquote}
        Für das Nachahmungs-Spiel werden insgesamt drei Spieler*innen benötigt. Ein Mann (A), eine Frau (B) und ein*e
        Fragesteller*in (C). Die Aufgabe des/der Fragesteller*in ist herrauszufinden wer von beiden die Frau ist.
        Dabei sitzt er/sie in einem anderen Raum als die beiden. Der/Die Fragesteller*in kennt beide Parteien
        nur unter X und Y, womit er am Ende des Spiel jeweils A und B identifiziert. Ziel von A ist es den/die Fragesteller*in
        fehlzuleiten. B verfolgt das Ziel, den/die Fragesteller*in zur richtigen Antwort zu leiten.
        \cite[433]{turing1950computing}
    \end{displayquote}
    In diesem Gedankenexperiement wird der Mann (A) nun durch eine Maschine ersetzt, die seine Aufgabe übernimmt und
    sich als Frau (B) ausgeben soll. Der Austausch von Informationen erfolgt dabei über maschinenschrift, damit
    der/die Fragesteller*in keine Schlussfolgerungen über Stimme oder Schrift ziehen kann.\cite[433]{turing1950computing}
    Bei diesem Test geht es nicht darum akutelle Maschinen zu betrachten und ein entgültigen Schluss zu ziehen. Es geht
    eher darum sich eine Maschine vorzustellen die diesen Test bestehen kann. Damit eine Maschine diesen Test besteht
    muss sie das Spiel mit einer 70\%iger Genauigkeit gewinnen.\cite[1]{oppy&dowe2020turingtest}


    \section{Das Chinesische Zimmer}
    Das Chinesische Zimmer ist ein Gedankenexperiment vom Philosophen John Searle welches
    versucht die Frage nach der erfolgreiche Entwicklung einer starken Künstlichen Intelligenz zu verneinen.
    Searles These ist, dass kein Computer jemals wie ein Menschen denken kann,
    obwohl sowohl der Computer als auch das Gehirn beides Systeme sind Symbole verarbeiten.\cite{nimtz2013chinesische}
    Dies begründet er mit folgendem Gedankenexperiment:
    \begin{displayquote}
        Stellt euch vor ich wäre in einem geschlossenen Raum mit einem großen Haufen chinesischer Texte.
        Ich kann weder Chinesisch sprechen noch lesen oder schreiben.
        Ebenfalls könnte ich chinesische von keiner anderen, wie beispielweise russischer, japanischer Schrift unterscheiden.
        Chinesische Schriftzeichen haben keine erkennbare Bedeutung und sind nur Formen für mich.

        Nun stellt euch vor ich würde einen zweiten Stapel erhalten. Dieser Stapel enthält weitere
        Chinesische Schriftzeichen sowie englische formale Regeln, die ich ohne Probleme verstehe.
        Diese formalen Regeln geben mir die Möglichkeit die chinesischen Schriftzeichen
        anhand ihrer Form zu identifizieren.

        Nun kriege ich einen dritten Stapel mit weiteren chinesischen Schriftzeichen und englisch Anweisungen die mir sagen wie ich
        diese neuen chinesischen Zeichen mit den Vorherigen vergleiche um bestimmte chinesische Zeichen zurückzugeben.

        Mit der Zeit werden die Leute außerhalb des Raumes immer besser mir Englische Anweisungen zu schreiben und
        ich werde immer besser diese auch zu verstehen, so dass meine Antworten ununterscheidbar von denen eines
        gebürtigen Chinesen werden. Doch verstehe ich, was ich an Chinesisch von mir gebe?
        \cite[1]{searle1999chinese}
    \end{displayquote}
    Searle versucht mit diesem Gedankenexperiment den Unterschied zwischen
    Syntaktik\footnote{Syntaktik (Syntax): Wie stellt man Zeichenketten zusammen, so dass sie Sinn ergeben?}
    und Semantik\footnote{Semantik: Was genau ist die Bedeutung hinter einem Wort?}
    greifbar zu machen.
    Ein Computer der keinerlei Verbindung in die Realität eines Menschen hat, kann zwar Regel lernen, die ihm die
    Welt der Menschen näher bringt, jedoch kann er niemals voll und ganz \emph{verstehen} oder auch \emph{begreifen},
    wie ein Mensch denkt. Damit setzt Searle eine eher negative Prognose auf den Forschritt, den die KI-Forschung
    machen wird oder besser gesagt, nicht machen wird.

    \chapter{Schlussbermerkung - Der Unterschied zwischen menschlicher und künstlicher Intelligenz}



    \printbibliography[
        title={Quellenverzeichnis},
        %heading=subbibintoc
    ]

\end{document}
