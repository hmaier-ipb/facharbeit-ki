%! Author = hmaier
%! Date = 09.09.21

% Preamble
\documentclass[12pt]{report}
\title{Künstliche Intelligenz - Ein Blick in die Zukunft}
\author{Hendrik Maier}
\date{}


% Packages
\usepackage[T1]{fontenc} % für die spezielle Quotierung, die mit
\usepackage{german}
\usepackage{titling}

% quotation-marks
\usepackage[
    left = \flqq{},% the special quote on the left (opening)
    right = \frqq{},% the special quote on the right (opening)
            ]{dirtytalk} % quoting
\usepackage{csquotes}

% bibliography
\usepackage{biblatex}
\addbibresource{ai_references.bib}



% Document
\begin{document}
    \maketitle

    \tableofcontents
    \newpage

    \section{Einführung: Was ist eigentlich Künstliche Intelligenz?}
    Im allgemeinen Sprachgebrauch beschreibt der Begriff \say{Künstliche Intelligenz} die Fähigkeit von Computer-Programmen,
    individulle Problemstellungen zu lösen. Die geschieht ohne die direkte Hilfe eines Technikers oder Programmieres.
    Mit \say{Intelligenz} ist dabei speziell gemeint, das vorher nie bearbeitete Probleme mithilfe ihrer Daten
    trotz ihrer Fremdheit zum Lösen eines Problems verwendet werden können. Die Fähigkeit der \say{Künstlichen Intelligenz}
    wird mithilfe des maschinellen Lernens erreicht. Grob gesagt kriegt ein Algorithmus (welcher als Computer-Programm geschrieben ist),
    eine Vielzahl von Daten eingespielt. Durch die Menge der Daten gelingt es dem Algorithmus, ein spezifisches
    Modell zu erstellen welches auf ähnliche aber jedoch unbekannte Daten angewendet werden kann.
    
    \section{Geschichte}

    \section{Starke versus Schwache KI}
    Die Idee eines mechanischen Helfer, der logische zu bearbeitende Aufgaben übernimmt, ist gar nicht so neu wie man
    zuerst vermuten würde. Wie auch andere bahnbrechende Erfindungen, werden die ersten Schritte auch bei dieser Idee
    mit einem Blatt Papier und etwas Tinte gegangen. Unter anderem Isaac Asimov hatte die Idee
    eines Roboters der sowohl als mechanischer Diener als auch als selbstdenkender Künstler agieren kann.\cite{asimov2000der}
    Mit dieser Idee, die nicht nur eine logisch agierende Maschine vorsieht, sondern auch ein denkendes Individuum, macht Asimov
    eine Teilung in zwei Kategorien die bis heute gilt. Die Rede ist von schwacher und starker Künstlicher Intelligenz.\\

    Als \emph{schwache Künstliche Intelligenz} bezeichnet man ein Großteil der heute eingesetzten Programme, die mit
    maschinellen Lernen trainiert worden sind.\cite{ibm2021whatisAI} Diese Art der KI erfüllt vordefinierte Aufgaben,
    wie beispielweise die Erkennung von Sprache oder Objekten. Dafür wird ein Vielzahl von vorbearbeiteten Beispielen der KI
    zum Lernen gegeben. Diese Beispiele sind vom Menschen auf eine Art und Weise bearbeitet so dass sie auf ein spezielles Ziel hindeuten.
    Der Mensch gibt der Maschine also ein Ziel so dass sie sich mit den vorgegebenen Daten beschäftigen kann.
    Ohne vorbestimmtes Ziel wäre es der Maschine nicht möglich die Daten zu deuten und zu verarbeiten.
    Endprodukt (tech. \say{Modell}) der Beschäftigung mit den Daten sind Regeln und Zusammenhänge
    mit denen die Problemstellung bearbeitet werden können. Ohne die Zuarbeit des Menschens, ist dieses Endprodukt nicht
    möglich, was bedeutet dass andere Probleme auf Grundlage der bisher eingepflegen Daten nicht zu lösen sind.
    Eine schwache KI kann also bestimmte trainierte Problemstellungen lösen, und dies sogar mit hoher Effizienz, doch bei
    unbekannten Parametern, versagen gelernte Regeln und Zusammenhänge.\\

    Um fremde unspezifizierte Problemstellungen zu Lösen, benötigt es einer \emph{starken Künstlichen Intelligenz}.
    Diese erweiterte Form der KI ist zum derzeitgen Zeitpunkt (Ende 2021) noch nicht realisiert worden und lässt sich am
    einfachsten mithilfe des \say{Turing Tests} definieren. Dieser Test wurde Mitte des 20. Jahrhunderts von
    Turing, einem britischen Mathematiker, erdacht und bespreibt folgendes Spiel:
    \begin{displayquote}
        Ein Mensch und ein Fragesteller werden in zwei seperierte
        Räume aufgeteilt. Ein Fragesteller, der keinen Sichtkontakt zu jeweils zu einem noch zum anderen der beiden Räume hat
        muss durchs Fragen herausfinden, wer von beiden der Mensch und wer der Computer ist.\cite{turing1950computing}
        Ziel des Computers ist den Fragenden irrezuleiten, so dass er glaubt dass der Computer der Mensch ist.
        Ziel der befragten Person ist es dem Fragenden bei der Identfikaton der Maschine zu helfen.\cite{turing1950computing}
    \end{displayquote}
    Falls es dem Computer gelingt den Fragenden irrezuleiten und ihn als Person zu identifieren, hat der Computer
    den \say{Turing Test} bestanden und gilt somit als denkfähigen Wesen was als starke Künstliche Intelligenz
    gleichzusetzen ist.\cite{oppy&dowe2020turingtest}






    \section{Einsatzgebiete und Anwendungsfälle}

    \subsection{Anwendungsbeispiele IPB}
    \section{Technische Grundlagen}
    \subsection{Supervised Learing}
    \subsection{Unsupervised Learning}
    \subsection{Linear Regression}
    \subsection{Logistic Regression}
    \subsection{Decision Tree}
    \subsection{Random Rorest Model}
    \subsection{Neural Networks (Deep Learning)}
    \section{Philosophische Betrachtung}
    In diesem abschließenden Abschnitt werde ich das Thema zusammenfassen sowie den Versuch unternehmen,
    Denkanstöße zu geben.
    \subsection{Intelligenzbegriff}
    In der Menschheitsgeschichte hatten Betrachtende bisher nur Tiere, Pflanzen oder andere Menschen zum Vergleich
    um den Begriff der \say{Intelligenz} zu definieren. Mit der vorranschreitenden Integration von Systemen die maschinelles
    Lernen nutzen um menschen-gemachte Aufgaben zu bewältigen, muss neu geklärt werden:
        \begin{displayquote}
            Was genau ist \say{Intelligenz}?
            Ab wann ist ein Wesen \say{intelligent}?
            Wie geht man mit weiter
        \end{displayquote}

    \printbibliography
\end{document}