%! Author = hmaier
%! Date = 09.09.21

% Preamble
\documentclass[12pt]{report}

\title{Künstliche Intelligenz - Ein Blick in die Zukunft}
\author{Hendrik Maier}
\date{}

% Packages
\usepackage[T1]{fontenc} % für die spezielle Quotierung
\usepackage{german}
\usepackage{titling}
\usepackage[
    left = \flqq{},% the special quote on the left (opening)
    right = \frqq{},% the special quote on the right (opening)
            ]{dirtytalk} % quoting
\usepackage{csquotes}
\usepackage{wasysym}
% Document
\begin{document}
    \maketitle

    \tableofcontents
    \newpage

    \section{Definition Begriffsklärung}
    \say{Künstliche Intelligenz} ist ein Kunstwort welches oft als Synonym für
    maschinelles Lernen benutzt wird.
    \section{Einsatzgebiete und Anwendungsfälle}
    \subsection{Anwendungsbeispiele IPB}
    \section{Technische Grundlagen}
    \subsection{Supervised Learing}
    \subsection{Unsupervised Learning}
    \subsection{Linear Regression}
    \subsection{Logistic Regression}
    \subsection{Decision Tree}
    \subsection{Random Rorest Model}
    \subsection{Neural Networks (Deep Learning)}
    \section{Philosophische Betrachtung}
    In diesem abschließenden Abschnitt werde ich das Thema zusammenfassen sowie den Versuch unternehmen,
    Denkanstöße zu geben.
    \subsection{Intelligenzbegriff}
    In der Menschheitsgeschichte hatten Betrachtende bisher nur Tiere, Pflanzen oder andere Menschen zum Vergleich
    um den Begriff der \say{Intelligenz} zu definieren. Mit der vorranschreitenden Integration von Systemen die maschinelles
    Lernen nutzen um menschen-gemachte Aufgaben zu bewältigen, muss neu geklärt werden:
        \begin{displayquote}
            Was genau ist \say{Intelligenz}?
            Ab wann ist ein Wesen \say{intelligent}?
            Wie geht man mit weiter
        \end{displayquote}


\end{document}