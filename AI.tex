%! Author = hmaier
%! Date = 09.09.21

% Preamble
\documentclass[12pt]{report}
\title{Künstliche Intelligenz - Ein Blick in die Zukunft?}
\author{Hendrik Maier}
\date{}


% Packages
\usepackage[T1]{fontenc} % für die spezielle Quotierung, die mit
\usepackage{german}
\usepackage{titling}
\usepackage{titlesec}

% quotation-marks
\usepackage[
    left = \flqq{},% the special quote on the left (opening)
    right = \frqq{},% the special quote on the right (opening)
            ]{dirtytalk} % quoting
\usepackage{csquotes}

% bibliography
\usepackage{biblatex}
\addbibresource{ai_references.bib}

% zeilenabstand
\usepackage[onehalfspacing]{setspace}

% Document
\begin{document}

    \begin{titlepage}
        \centering
        {\scshape\LARGE Leibniz Institut für Pflanzenbiochemie\par}
        \vspace{1cm}
        {\scshape\Large Fachbericht September 2021\par}
        \vspace{1.5cm}
        {\huge\bfseries Künstliche Intelligenz - Ein Blick in die Zukunft?\par}
        \vspace{2cm}
        {\Large Abteilung: Geräte \& IT-Service\par}
        \vspace{1.5cm}
        {\Large\itshape Hendrik Maier\par}
        \vfill

% Bottom of the page
        % {\large \today\par}
    \end{titlepage}

    \tableofcontents
    \newpage

    \section{Einführung: Was ist eigentlich Künstliche Intelligenz?}
    Um eine kurze Einleitung in das Thema zu geben, möchte ich vorerst Begriffsklärung des zu behandelnden Begriffs,
    \say{Künstliche Intelligenz}, vornehmen um zu verstehen was dieser genau bedeutet.
    Der Begriff \emph{Intelligenz} kommt vom lateinischen \say{intelligere}. Dies bedeutet Einsehen, Begreifen und Erkennen.\cite{piaget2000psychologie}
    \say{Künstlich} verweist dabei auf die unnatürliche Herkunft der Einsicht. In Kombination beschreibt \say{künstliche Intelligenz}
    also ein Einsehen welches nicht der natürlichen Art entspricht. Der Rahmen dieses Fachberichts wird sich mit Erklärungen
    der Geschichte und der Technik hinter KI gefüllt werden. Der technische Teil wird sich am meisten
    mit der Programmierung und dem Ablauf bei der Erschaffung befassen. Die Mathematik die sich hinter
    der Programmierung befindet wird ebenfalls nur oberflächlich behandelt werden.
    
    \section{Enstehungsgeschichte}
    Wie bei vielen neuzeitlichen Erfindungen wurde auch die Forschung an \say{Künstliche Intelligenz}
    erstmalig von verschiedenen Denkern und Schriftstellern angestoßen.
    Nicht erst Science-Fiction Autoren wie Isaac Asimov oder Jule Vernes haben die Idee von intelligenten Maschinen
    entwickelt, sondern schon der Grieche Homer schrieb von mechanischen Dienern die den Göttern beim Abendsessen
    Wein nachschenkten.\cite{buchanan2005very} Auch wenn diese Verwendung von KI aus unserem heutigen Standpunkt
    eher banal erscheint, ist die Funktionsweise aus damaligen Sicht unerklärbar.
    Ein wenig weiter dachte der Philosoph Gottfried Wilhelm Leibniz, der über mechanische Richter nachdachte
    die aufgrund von logischen Regeln Rechtsfälle aushandeln.\cite{buchanan2005very}
    Dieses Beispiel stößt schon ziemlich nah an die Vorstellung von künstlicher Intelligenz die wir heutzutage haben.
    Was beide Beispiele jedoch gemeinsam haben ist dass keiner von beiden ihren Apparaten eigenes Denken
    gibt. Sie werden lediglich als logisch operierende Maschinen angesehen, die ohne den Menschen nicht wissen würden
    was sie tun sollten.
    Um die Möglichkeit in Erwägung zu ziehen, ob Maschinen denken könnten, brauchte es erst den Mathematiker Alan Turing
    der dies Mitte des 20. Jahrunderts diskutierte.\cite{sesink1993menschliche}


    \section{Arten der Künstlichen Intelligenz - Starke versus Schwache KI}
    Die Idee eines mechanischen Helfer, der logische zu bearbeitende Aufgaben übernimmt, ist gar nicht so neu wie man
    zuerst vermuten würde. Wie auch andere bahnbrechende Erfindungen, werden die ersten Schritte auch bei dieser Idee
    mit einem Blatt Papier und etwas Tinte gegangen. Unter anderem Isaac Asimov hatte die Idee
    eines Roboters der sowohl als mechanischer Diener als auch als selbstdenkender Künstler agieren kann.\cite{asimov2000der}
    Mit dieser Idee, die nicht nur eine logisch agierende Maschine vorsieht, sondern auch ein denkendes Individuum, macht Asimov
    eine Teilung in zwei Kategorien die bis heute gilt. Die Rede ist von schwacher (logisch agierender) und starker (denkender) Künstlicher Intelligenz.\\

    Als \emph{schwache Künstliche Intelligenz} bezeichnet man ein Großteil der heute eingesetzten Programme, die mit
    maschinellen Lernen trainiert worden sind.\cite{ibm2021whatisAI} Diese Art der KI erfüllt vordefinierte Aufgaben,
    wie beispielweise die Erkennung von Sprache oder Objekten. Dafür wird ein Vielzahl von vorbearbeiteten Beispielen der KI
    zum Lernen gegeben. Diese Beispiele sind vom Menschen auf eine Art und Weise bearbeitet so dass sie auf ein spezielles Ziel hindeuten.
    Der Mensch gibt der Maschine also ein Ziel so dass sie sich mit den vorgegebenen Daten beschäftigen kann.
    Ohne vorbestimmtes Ziel wäre es der Maschine nicht möglich die Daten zu deuten und zu verarbeiten.
    Endprodukt (tech. \say{Modell}) der Beschäftigung mit den Daten sind Regeln und Zusammenhänge
    mit denen die Problemstellung bearbeitet werden können. Ohne die Zuarbeit des Menschens, ist dieses Endprodukt nicht
    möglich, was bedeutet dass andere Probleme auf Grundlage der bisher eingepflegen Daten nicht zu lösen sind.
    Eine schwache KI kann also bestimmte trainierte Problemstellungen lösen, und dies sogar mit hoher Effizienz und Genauigkeit, doch bei
    unbekannten Parametern, versagen gelernte Regeln und Zusammenhänge.\\

    Um fremde unspezifizierte Problemstellungen zu Lösen, benötigt es einer \emph{starken Künstlichen Intelligenz}.
    Diese erweiterte Form der KI ist zum derzeitgen Zeitpunkt (Ende 2021) noch nicht realisiert worden und lässt sich am
    einfachsten mithilfe des \say{Turing Tests} definieren. Dieser Test wurde Mitte des 20. Jahrhunderts von
    Turing, einem britischen Mathematiker, erdacht und bespreibt folgendes Spiel:
    \begin{displayquote}
        Ein Mensch und ein Fragesteller werden in zwei seperierte
        Räume aufgeteilt. Ein Fragesteller, der keinen Sichtkontakt zu jeweils zu einem noch zum anderen der beiden Räume hat
        muss durchs Fragen herausfinden, wer von beiden der Mensch und wer der Computer ist.
        Ziel des Computers ist den Fragenden irrezuleiten, so dass er glaubt dass der Computer der Mensch ist.
        Ziel der befragten Person ist es dem Fragenden bei der Identfikaton der Maschine zu helfen.\cite{turing1950computing}
    \end{displayquote}
    Falls es dem Computer gelingt, den Fragenden irrezuleiten und ihn (den Computer) als Person zu identifieren, hat der Computer
    den \say{Turing Test} bestanden und gilt somit als denkfähigen Wesen was als \emph{starke Künstliche Intelligenz}
    bezeichnet wird.\cite{oppy&dowe2020turingtest}. Ein solches denkfähiges Wesen besitzt die Fähigkeit verschiedene Problemstellungen
    auf kreative Art und Weise zu lösen, da es nicht wie ein klassischer Computer fest auf ein Thema trainiert ist sondern
    sich flexibel selber(!) Gedanken machen kann.\\

    Hier zeigt sich nun auch der genaue Unterschied zwischen schwacher und starker Künstlicher Intelligenz: eine Machine,
    die auf Grund gelernter Regeln Probleme lösen kann ist \emph{schwach}. Eine Maschine die jedoch ebenfalls selbst denken
    kann ist \emph{stark}.


    \section{Einsatzgebiete und Anwendungsfälle}
    \subsection{Anwendungsbeispiele IPB}

    \section{Technische Grundlagen}
    \subsection{Machinelles Lernen}
    \subsubsection{Supervised Learing}
    \subsubsection{Unsupervised Learning}
    \subsubsection{Linear Regression}
    \subsubsection{Logistic Regression}
    \subsubsection{Decision Tree}
    \subsubsection{Random Rorest Model}

    \subsection{Deep Learning}
    \subsubsection{Neural Networks (Deep Learning)}
    \section{Gedankenexperimente}
    \subsection{Turing Test}
    \subsection{Chinesischer Raum}


    \printbibliography
\end{document}